\section{Rappels sur le noyau fonctionnel}

\begin{frame}
  \begin{center}
    \frametitle{la programmation fonctionnel}
  Le noyau fonctionnel, la sup, le bon temps, toussa toussa....
  \end{center}
\end{frame}

\subsection{Les nombres}
\begin{frame}
  \frametitle{le noyaux fonctionnel}
  \framesubtitle{Les nombres}
  \begin{center}
    les nombres ...\\
   \begin{minipage}[t]{5cm}
    les ints : \\
    \#42;;\\
    - : int  = 42\\
   \end{minipage}
   \begin{minipage}[t]{5cm}
    les floats :\\
    \#4.2;;\\
    - : float = 4.2
   \end{minipage}
   \vspace{0.5cm}
    et leurs utilisations\\
   \begin{minipage}[t]{5cm}
    \#4 * 10 + 2;;\\
      - : int = 42
   \end{minipage}
   \begin{minipage}[t]{5cm}
     \#41.5 +. 0.5;;\\
     - : float = 42
   \end{minipage}
  \end{center}
\end{frame}

  \subsection{les chars et chaines de chars}
  \begin{frame}
    \frametitle{le noyaux fonctionnel}
    \framesubtitle{caractères et chaines de caractère}
    Les caractères sont bien pratique si l'on veux faire un interfacage avec l'utilisateur ou d'autre ... truc (*caugh* ocr *caugh*)\\
    \begin{minipage}[t]{10cm}
      \#"coucou moi aussi je post sur JVC";;\\
      - : string = "coucou moi aussi je post sur JVC"
    \end{minipage}
    \begin{minipage}[t]{5cm}
      \#'4';;\\
      - : char = '4'
    \end{minipage}
  \end{frame}

  \subsection{les boolèens}
  \begin{frame}
    \frametitle{le noyaux fonctionnel}
    \framesubtitle{les boolèens}
      la logique booléenne est composé de 2 "termes" true et false qui s'utilise avec les operateur suivant :
      \begin{center}
      \begin{minipage}[t]{7cm}
	  \begin{itemize}
	    \item "not" : negation
	    \item "\&\&" : et sequentiel
	    \item "||" : ou sequentiel
	    \item "\&" : synonyme pour "\&\&"
	  \end{itemize}
	\end{minipage}
      \end{center}
	\begin{minipage}[t]{5cm}
	  \begin{itemize}
	    \item "=" : egalité structurelle
	    \item "==" : egalité physique
	    \item "<>" : negation de "="
	    \item "!=" : negation de "=="
	  \end{itemize}
	\end{minipage}
	\begin{minipage}[t]{5cm}
	  \begin{itemize}
	    \item "<" : inferieur 
	    \item ">" : superieur
	    \item "<=" : inferieur égale
	    \item ">=" : superieur égale
	  \end{itemize}
	\end{minipage}
    \end{frame}

    \subsection{le controle conditionnel}
    \begin{frame}
      \frametitle{le noyaux fonctionnel}
      \framesubtitle{le controle conditionnel}
      if a then b else c;;
    \end{frame}

    \subsection{les declarations}
\begin{frame}[fragile]
      \frametitle{le noyaux fonctionnel}
      \framesubtitle{les declarations}
      il existe deux types de declaration :
     \begin{minipage}[t]{5cm}
	les globales :\\
	\begin{lstlisting}
	let a = "42";;
	val a : string = 42
	\end{lstlisting}
	avec repetition :
	\begin{lstlisting}
	let a = 2
	and b = 3
	and c = 4;;
	val a : int = 2
	val b : int = 3
	val c : int = 4
	\end{lstlisting}
      \end{minipage}
      \begin{minipage}[t]{5cm}
	les locales :\\
	\begin{lstlisting}
	let a = 2 in a + 1;;
	- : int = 3
	\end{lstlisting}
	avec repetition :
	\begin{lstlisting}
	let a = 2
	and b = 3
	and c = 4 in
	a + b + c;;
	- : int = 9
	\end{lstlisting}
      \end{minipage}
\end{frame}

    \subsection{Les fonctions}
    \begin{frame}
      \frametitle{Le noyaux fonctionnel}
      \framesubtitle{les fonctions}
      \#function x -> x+1;;
      -: int -> int = <fun>
    \end{frame}

      \subsection{le filtrage de motif}
      \begin{frame}
      \end{frame}

      \subsection{Les types}
      \begin{frame}
      \end{frame}
