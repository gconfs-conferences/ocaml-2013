\section{Rappels des bases d'OCaml}
\begin{frame}
  \begin{center}
	\huge
	La base, la sup, etc... et deux, trois, autres nouveautés !
  \end{center}
\end{frame}

\subsection{Les données de base} %%%%%%%%%%%%%%%%%%%%%%%%%%%%%%%
\begin{frame}[fragile]
	\frametitle{Les types élémentaires}
	\begin{itemize}
		\item Les entiers : 42\\
			~~4 * 10 + 2
		\item Les flottants : 4.2\\
			~~41.5 +. 0.5
		\item Les caractères : 'c'\\
		\item Les chaînes de caractères : "abcd"\\
			~~"The " \^~"meaning of " \^~"life"
			\item Les n-uplets : ( 1, 4.2, "abc", 4 )
		\item Le unit : ()
	\end{itemize}
\end{frame}

\begin{frame}[fragile]
	\frametitle{Les listes}
	\framesubtitle{Mais qu'est-ce que c'est ?}
	\begin{itemize}

	\item Liste vide:
		\begin{lstlisting}
		[]
		\end{lstlisting}

	\item Opérateur de construction:
		\begin{lstlisting}
		a :: []
		1 :: 2 :: 3 :: 4 :: 5 :: []
		"abc" :: "de" :: "f" :: "ghij" :: []
		\end{lstlisting}

	\item Concaténation (à éviter):
		\begin{lstlisting}
		liste1 @ liste2
		\end{lstlisting}

	\end{itemize}
\end{frame}

\begin{frame}[fragile]
	\frametitle{Les listes}
	\framesubtitle{Quelques fonctionnalités du module List}
	\begin{itemize}
	
	\item List.length : retourne la longueur d'une liste
	
	\item List.append : autre nom de l'opérateur @
	
	\item List.rev : inverse la liste passée en paramètre
	
	\item List.iter : application d'une fonction f à chacun des éléments de la liste

	\item List.map : renvoie une liste des résultats de l'application d'une fonction f à chacun des éléments de la liste

	\end{itemize}
\end{frame}


\subsection{Déclaration et fonctions} %%%%%%%%%%%%%%%%%%%%%%%%%%
\begin{frame}[fragile]
      \frametitle{Les déclarations}
      Il existe deux types de déclaration :
	\begin{columns}[t]
		\begin{column}{5cm}
		\begin{block}{Les globales}
		\begin{lstlisting}
	let a = "42";;
	val a : string = "42"
	\end{lstlisting}
	avec répétition :
	\begin{lstlisting}
	let a = 2
	and b = 3
	and c = 4;;
	val a : int = 2
	val b : int = 3
	val c : int = 4
		\end{lstlisting}
		\end{block}
		\end{column}
      		\begin{column}{5cm}
		\begin{block}{Les locales}
		\begin{lstlisting}
	let a = 2 in a + 1;;
	- : int = 3
	\end{lstlisting}
	avec répétition :
	\begin{lstlisting}
	let a = 2
	and b = 3
	and c = 4 in
	a + b + c;;
	- : int = 9
		\end{lstlisting}
		\end{block}
      		\end{column}
	\end{columns}
\end{frame}

\begin{frame}[fragile]
	\frametitle{Les fonctions}
	\framesubtitle{Représentation}
	\begin{center}
		\begin{minipage}{10cm}
				\begin{lstlisting}
     function x -> x + 1;;
     -: int -> int = <fun>
				\end{lstlisting}
				\begin{lstlisting}
     function x -> function y -> x + y;;
     - : int -> int -> int = <fun>
				\end{lstlisting}
				\begin{lstlisting}
     function (x , y) -> x * y;;
     - : int * int -> int = <fun>
				\end{lstlisting}
				\begin{lstlisting}
     fun x y -> x + y;;
     - : int -> int -> int = <fun>
				\end{lstlisting}
		\end{minipage}
  \end{center}
\end{frame}

\begin{frame}[fragile]
	\frametitle{Les fonctions}
  	\framesubtitle{Déclaration}
    	\begin{lstlisting}
	let name p1 pn = expr
    	\end{lstlisting}
	\begin{lstlisting}
let name =
  function p1 -> function pn -> expr
  	\end{lstlisting}
  	\vspace{0.4cm}
  	\begin{lstlisting}
let add x y = x + y;;
val add : int -> int -> int = <fun>
  	\end{lstlisting}
\end{frame}

\begin{frame}[fragile]
	\frametitle{Les fonctions}
	\framesubtitle{Déclaration récursive}
	\begin{block}{Déclaration de fonction récursive}
	  \begin{lstlisting}
	  let rec facto x =
	  	if x = 0 then 1 else x * facto(x-1);;
	  val facto : int -> int = <fun>
	  \end{lstlisting}
	\end{block}
\end{frame}

\begin{frame}[fragile]
	\frametitle{Les fonctions}
  	\framesubtitle{L'ordre supérieur}
  	Et si nous faisions des fonctions de fonctions ?
 	\begin{lstlisting}
  let comp f a b = f a b;;
  val comp : ('a -> 'b -> 'c) ->
  'a -> 'b -> 'c = <fun>

  comp (+) 3 2;;
  - : int = 5
 	\end{lstlisting}
	Nous retrouvons ici le type polymorphe, rassemblant tous les\\
	les types et representé par les 'a et 'b.
\end{frame}

\begin{frame}[fragile]
	\frametitle{Les fonctions}
	\framesubtitle{Les paramètres optionnels}	

	 	\begin{block}{Déclaration}
		\begin{lstlisting}
		  let incr ?step:(x = 1) r = r + x
		
		  val incr : ?step:int -> int -> int
		\end{lstlisting}
		\end{block}

	 	\begin{block}{Utilisation}
		\begin{lstlisting}
		  incr 32;;
		  - : int = 33
		  incr ~step:10 32
		  - : int = 42
		\end{lstlisting}
		\end{block}

\end{frame}

\subsection{Faire des choix} %%%%%%%%%%%%%%%%%%%%%%%%%%%%%%%%%%%
\begin{frame}[fragile]
	\frametitle{Le contrôle conditionnel}
	\begin{block}{La structure}
	\begin{lstlisting}
	  let f a b c =
	    if a
	     then b
	     else c

	  val f : bool -> 'a -> 'a -> 'a
	\end{lstlisting}
	\end{block}
	\begin{block}{'else' optionnel pour le type de retour unit}
	\begin{lstlisting}
	  let f a =
	    if a=42
	      then print_string "42!"
	      (* else () *)

	  val f : bool -> unit -> unit -> unit
	\end{lstlisting}
	\end{block}
\end{frame}

\begin{frame}[fragile]
    \frametitle{Les tests booléens}
	\begin{columns}
		\begin{column}{5cm}
			\begin{itemize}
				\item "not" : négation
				\item "\&\&" : ET logique
				\item "||" : OU logique
				\item "=" : egalité structurelle
				\item "<>" : négation de "="
			\end{itemize}
		\end{column}
		\begin{column}{5cm}
			\begin{itemize}
				\item "<" : inférieur strict
				\item ">" : supérieur strict
				\item "<=" : inférieur ou égal
				\item ">=" : supérieur ou égal
			\end{itemize}
		\end{column}
	\end{columns}
\end{frame}

\begin{frame}[fragile]
	\frametitle{Le filtrage de motif}
	\framesubtitle{Généralités}
	\begin{block}{La structure}
		\begin{lstlisting}
  match expr with
     m1 -> expr1
   | m2 -> expr2
   | mn -> exprn
		\end{lstlisting}
	\end{block}
	\begin{block}{Exemples}
		\begin{columns}
		\begin{column}{5cm}
		\begin{lstlisting}
  match i with
     0 -> expr1
   | 5 | 12 -> expr2
   | n -> exprn
		\end{lstlisting}
		\end{column}
		\begin{column}{5cm}
		\begin{lstlisting}
  function
      42 -> true
    | _ -> false
		\end{lstlisting}
		\end{column}
	\end{columns}
	\end{block}
\end{frame}

\begin{frame}[fragile]
	\frametitle{Le filtrage de motif}
	\framesubtitle{Récursion et listes}
		\begin{lstlisting}
  let rec printStrList l = match l with
      [] -> print_string "\n"
    | e::[] -> print_string (e ^ "\n")
    | e::q ->
      print_string(e ^ ", ");
      printStrList q
		\end{lstlisting}
\end{frame}

\begin{frame}[fragile]
  	\frametitle{Le filtrage de motif}
	\framesubtitle{'as' et 'when'}
  	\begin{block}{As}
    	\begin{lstlisting}
	  let rec fibo = function
 	    0 | 1 as n -> n
 	  | x -> fibo(x - 1) + fibo(x - 2);;
    	\end{lstlisting}
  	\end{block}
  	\begin{block}{When}
    	\begin{lstlisting}
  let is_inf = function
        x when x < 42 -> true
    	| _ -> false;;
    	\end{lstlisting}
  	\end{block}
\end{frame}

\subsection{Les types} %%%%%%%%%%%%%%%%%%%%%%%%%%%%%%%%%%%%%%%%%
\begin{frame}[fragile]
	\frametitle{Créer un type}
	\framesubtitle{Types simples}
	\begin{lstlisting}
  type name = typeToDefine

  type point = int * int

  let a = (3,7);;
  val a : int * int = (3,7)
  let (b:point) = (3,7);;
  val b : point = (3,7)

  type stringLabeledIntList =
    string * (int list)

  type funIntList = (int list) -> int
	\end{lstlisting}
\end{frame}

\begin{frame}[fragile]
	\frametitle{Créer un type}
	\framesubtitle{Types paramétrés}
	\begin{lstlisting}
  type ('nameTypeParameter1,
 	     'nameTypeParameterN)
   	    name =
         typeToDefine

  type ('valType) labeledList =
    string * ('valType list)
	\end{lstlisting}
\end{frame}


\begin{frame}[fragile]
	\frametitle{Créer un type}
	\framesubtitle{Types somme}
	\begin{columns}
		\begin{column}{5cm}
			\begin{lstlisting}
  type head =
    | King
    | Queen
    | Jack

  type cardFigure =
    | Ace
    | NumberCard of int
    | HeadCard of head
			\end{lstlisting}
		\end{column}
		\begin{column}{7cm}
			\begin{lstlisting}
  type cardColor =
    | Spade
    | Heart
    | Diamond
    | Club

  type card =
    cardColor * cardFigure
			\end{lstlisting}
		\end{column}
	\end{columns}
\end{frame}


\begin{frame}[fragile]
	\frametitle{Créer un type}
	\framesubtitle{Types enregistrements}
	\begin{itemize}
	\item Déclaration de type:
		\begin{lstlisting}
		  type pixel =
		  {
		     mutable r:float;
		     mutable g:string;
		     b:int;
		  }

		let p = { r=2.4; g="The Game"; b=255 }
		\end{lstlisting}
	\item Accès aux champs et modification de leur valeur :
		\begin{lstlisting}
			  p.b;;
			  -: int = 255
			  p.g <- "You Lose"
		\end{lstlisting}
	\end{itemize}
\end{frame}



\begin{frame}[fragile]
	\frametitle{Les exceptions}
	\begin{lstlisting}
 exception Help
 exception HelpNbErr of int
	\end{lstlisting}
	\begin{block}{Pour lever une exception :}
		\begin{lstlisting}
  raise Help
  raise (HelpNbErr 451)

  invalid_arg "element parameter"
  failwith "uninitialized Graphics"
		\end{lstlisting}
	\end{block}
	\begin{block}{Pour traiter une exception}
		\begin{center}
			\begin{minipage}{4.2cm}
				\lstset{basicstyle=\scriptsize}
				\begin{lstlisting}
try expr with
| Exc1 -> expr1
| Exc2 (v1,v2) -> expr2
| Excn -> exprn
				\end{lstlisting}
			\end{minipage}
			\begin{minipage}{5cm}
				\lstset{basicstyle=\scriptsize}
				\begin{lstlisting}
try expr with
| MyExc -> print_string "lol"
| Invalid_argument n ->
    print_string ("InvArg:"^n)
| Failure -> exit (-1)
				\end{lstlisting}
			\end{minipage}
		\end{center}
	\end{block}
\end{frame}
