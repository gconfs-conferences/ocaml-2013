\section{Rappels sur le noyau fonctionnel}

\begin{frame}
  \begin{center}
  Le noyau fonctionnel, la sup, le bon temps, toussa toussa....
  \end{center}
\end{frame}

\subsection{Les nombres}
\begin{frame}
  \begin{center}
  Que serait un langage sans nombre, sans quantification ? 
  \end{center}
  Il existe donc deux "types" de nombres en Ocaml (bien plus enfaite mais c'est un mensonge) qui sont ... Les Int et les Floats !\\
  \vspace{0.5cm} 
  \begin{minipage}[t]{5cm}
  \#42;;\\
  - : int  = 42
\end{minipage}
\begin{minipage}[t]{5cm}
\#4.2;;\\
- : float = 4.2
\end{minipage}

\end{frame}


  \subsubsection{Les operation}
  \begin{frame}
    Une fois que l'on a ces nombres il nous faut donc les manipuler. 
    operation chez les Int : "+" : addition / "-" : soustraction / "/" : division / "*" : multiplication\\
    \begin{center}
    C'est chiant les bases non ?\\ 
  \end{center}
    Aller on vous epargne les floattant c'est la même chose mais on rajoute juste un point aprés le symbole (ex : *.)\\
    \vspace{0.5cm}
    \begin{minipage}[t]{5cm}
      \#4 * 10 + 2;;\\
      - : int = 42
     \end{minipage}
     \begin{minipage}[t]{5cm}
       \#41.5 +. 0.5;;\\
       - : float = 42
     \end{minipage}
   \end{frame}
  \subsection{on enchaine les caractères pour les utiliser}
  \begin{frame}
    Les caractères sont bien pratique si l'on veux faire un interfacage avec l'utilisateur ou d'autre ... truc (*caugh* ocr *caugh*)
    \vspace{0.5cm}
    \begin{minipage}[t]{10cm}
      \#"coucou moi aussi je post sur JVC";;\\
      - : string = "coucou moi aussi je post sur JVC"
    \end{minipage}
    \begin{minipage}[t]{5cm}
      \#'4';;\\
      - : char = '4'
    \end{minipage}
     
  \end{frame}
