\section{Le noyau imperatif}

\begin{frame}
quoi de neuf ? \\
the ;; is a lie !
\end{frame}

%\begin{frame}[fragile]
%  \frametitle{le noyaux imperatif}
%  \framesubtitle{les entrées-sorties}
%  les canaux : 
%  \begin{lstlisting}
%      open_in
%      open_out
%      close_in
%      close_out
%   \end{lstlisting}
%    ecriture et lecture :
%  \begin{minipage}[t]{4cm}
%   \begin{lstlisting}
%    input_line
%    intput
%    output
%   \end{lstlisting}
%  \end{minipage}
%  \begin{minipage}[t]{4cm}
%   \begin{lstlisting}
%    print_newline
%    print_string
%    read_line
%   \end{lstlisting}
%  \end{minipage}
%\end{frame}

\begin{frame}[fragile]
  \frametitle{le noyau imperatif}
  \framesubtitle{le sequencage}
  separer des expressions :
  \begin{lstlisting}
  expr1;...;expr2
  \end{lstlisting}
  faire des "blocs" de code :
  \begin{lstlisting}
  begin
    expr1; 
    .
    .
    .
    exprn
  end
  \end{lstlisting}
\end{frame}

\begin{frame}[fragile]
    \frametitle{le noyau imperatif}
    \framesubtitle{les boucles}
    \begin{itemize}
      \item
	la boucle for :
	\begin{lstlisting}
	for nom = expr1 to expr2 do 
	expr3
	done

	for nom = expr1 downto expr2 do
	expr3
	done
	\end{lstlisting}
      \item
	la boucle while :
	\begin{lstlisting}
	while expr1 < expr2 do
	expr3
	done
      \end{lstlisting}
  \end{itemize}
\end{frame}
