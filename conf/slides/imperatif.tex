\subsection{Le côté imperatif d'OCaml} %%%%%%%%%%%%%%%%%%%%%%
%\begin{frame}[fragile]
%  \frametitle{le noyaux imperatif}
%  \framesubtitle{les entrées-sorties}
%  les canaux : 
%  \begin{lstlisting}
%      open_in
%      open_out
%      close_in
%      close_out
%   \end{lstlisting}
%    ecriture et lecture :
%  \begin{minipage}[t]{4cm}
%   \begin{lstlisting}
%    input_line
%    intput
%    output
%   \end{lstlisting}
%  \end{minipage}
%  \begin{minipage}[t]{4cm}
%   \begin{lstlisting}
%    print_newline
%    print_string
%    read_line
%   \end{lstlisting}
%  \end{minipage}
%\end{frame}

\begin{frame}[fragile]
	\frametitle{Le sequençage}
		\begin{block}{Séparer des expressions}
		\begin{lstlisting}
  expr1;...;expr2
		\end{lstlisting}
		\end{block}
		\begin{block}{Faire des "blocs" de code}
		\begin{lstlisting}
    begin
    expr1; 
    .
    .
    .
    exprn
    end
		\end{lstlisting}
		\end{block}
\end{frame}

\begin{frame}[fragile]
    \frametitle{Les boucles}
    \begin{block}{La boucle for}
	\begin{lstlisting}
	for nom = expr1 to expr2 do 
	expr3
	done

	for nom = expr1 downto expr2 do
	expr3
	done
	\end{lstlisting}
      \end{block}
	\begin{block}{La boucle while}
	\begin{lstlisting}
	while expr1 < expr2 do
	expr3
	done
      \end{lstlisting}
  \end{block}
\end{frame}
