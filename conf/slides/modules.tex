\section{Organisation d'un projet}

\begin{frame}
	\begin{center}
		\huge
		Organisation d'un projet
	\end{center}
\end{frame}

\subsection{Les modules} %%%%%%%%%%%%%%%%%%%%%%%%%%%%%%%%%%%%%%%%%%%%%%%%%%%%%%
\begin{frame}
	\frametitle{Principe des modules}
	
\end{frame}

\begin{frame}
	\frametitle{Utilisation de fonctions d'un module}

\end{frame}

\begin{frame}
	\frametitle{Une facon de creer un module : Un fichier...}

\end{frame}

\begin{frame}
	\frametitle{Les fichiers interface}

\end{frame}

\begin{frame}
	\frametitle{Syntaxe d'un fichier interface}

\end{frame}

\subsection{Faire de la documentation} %%%%%%%%%%%%%%%%%%%%%%%%%%%%%%%%%%%%%%%%
\begin{frame}
	\frametitle{La documentation}

\end{frame}

\begin{frame}
	\frametitle{ocamldoc}

\end{frame}

\begin{frame}
	\frametitle{La position des commentaires}

\end{frame}

\subsection{La compilation avec ocamlbuild} %%%%%%%%%%%%%%%%%%%%%%%%%%%%%%%%%%%
\begin{frame}
	\frametitle{Les differentes sorties d'OCaml}

\end{frame}

\begin{frame}
	\frametitle{Un Makefile rustique}

\end{frame}

\begin{frame}
	\frametitle{ocamlbuild}

\end{frame}

\begin{frame}
	\frametitle{Inclure des bibliotheques}

\end{frame}
