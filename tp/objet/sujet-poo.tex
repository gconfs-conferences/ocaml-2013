%
% Theophile Ranquet <ranquet@lrde.epita.fr>
% dim. oct.  6 05:52:34 CEST 2013
%

\documentclass[a4paper]{article}
\usepackage[utf8]{inputenc}
\usepackage[T1]{fontenc}
\usepackage[french]{babel}
\usepackage{fixltx2e}
\usepackage{nth}
\usepackage{listings}
\usepackage{tikz}
\usepackage{fancyhdr}
\usepackage[margin=1.0in]{geometry}
\usepackage{hyperref}

\pagestyle{fancy}
\lhead{{\sc{OCaml}}\\ {\sc TP} Objet - 11 oct. 2013}
\rhead{{\small \sc{GConfs}}\\ {\sc Epita}}
\rfoot{\includegraphics[width=0.15\linewidth]{gconfs.png}}

\begin{document}
\begin{center}
  {\Large {\bf La Programmation Orientée Objet en
  \textsc{OCaml}\protect\footnote{Questions, remarques, et plaintes à
\url{ranquet@lrde.epita.fr}, \protect\url{yroeht@irc.rezosup.org}}}}
\end{center}

\bigskip

\section*{Introduction\protect\footnote{\url{http://fr.wikipedia.org/wiki/OCaml\#Orient.C3.A9\_objet}}}

\textsc{OCaml} se distingue particulièrement par son extension du typage
\textsc{ML} vers un système objet comparable à ceux utilisés par les langages
objets classiques.  Cela permet un sous-typage structurel, dans lequel les
objets sont de types compatibles si les types de leurs méthodes sont
compatibles, indépendamment de leurs arbres d'héritage respectifs. Cette
fonctionnalité, que l'on peut considérer comme l'équivalent du \textit{duck
typing} des langages \textbf{dynamiques}, permet une intégration naturelle des
concepts objets dans un langage globalement fonctionnel.

Ainsi, à la différence des langages orientés objet tels \textsc{C++} ou
\textsc{Java} pour lesquels toute classe définit un type, les classes
\textsc{OCaml} définissent plutôt des abréviations de types. En effet, pour peu
que le type des méthodes soient compatibles, deux objets de deux classes
différentes peuvent être utilisés indifféremment dans un même contexte. Cette
caractéristique de la couche objet de \textsc{OCaml} rompt bon nombre de
principes communément admis : \textbf{il est en effet possible de faire du
sous-typage sans héritage}, par exemple. Le côté polymorphe rompt le principe
inverse. Des exemples de code, bien que rares, exhibant des cas d'héritage sans
sous-typage existent
également.

\section{I Can Win}

\subsection{Classe, objet, et méthode}

Ecrivez la \textbf{classe} \verb|cat|, dotée d'une \textbf{méthode} \verb|meow|
pour faire miauler votre chat.

\vspace{3mm}
\begin{verbatim}
  class cat : object
    method meow : unit -> unit
  end
\end{verbatim}
\vspace{3mm}

Instanciez un chat Dinah\footnote{Dinah est le chat d'Alice, dans \textit{Alice
au pays des merveilles}}, et faites le miauler. Par exemple, le code suivant:

\vspace{3mm}
\verb|  let dinah = (* FIXME : Some code was deleted here. *) in dinah#meow()|
\vspace{3mm}

Affichera:

\vspace{3mm}
\verb|  Meeeow!|
\vspace{3mm}

\subsection{Attribut}

Ajoutez quatre pattes (cet \textbf{attribut} est une valeur mutable initialisée
à 4) à votre class \verb|cat|, ainsi qu'une méthode \verb|break_a_leg| qui
décrémente ce nombre de pattes.

La nouvelle
interface de votre classe est:
\vspace{3mm}
\begin{verbatim}
  class cat : object
    val legs : int
    method meow : unit -> unit
    method break_a_leg : unit -> unit
  end
\end{verbatim}
\vspace{3mm}

Utilisez une boucle (dans votre méthode \verb|meow|) pour que votre chat
communique\footnote{\protect\url{http://en.wikipedia.org/wiki/Cat\_communication}}
désormais en miaulant une fois par patte.

\newpage

Vous pouvez tester avec ceci:

\begin{verbatim}
let cats = Array.init 3 (fun _ -> new cat) in
Array.iter (fun c -> if Random.int 2 == 0 then c#break_a_leg ()) cats;
Array.iter (fun c -> c#meow (); print_newline()) cats
\end{verbatim}

Vous devriez constater que certains chats miaulent une fois de moins que les
autres. Vous pouvez, selon votre degré de cruauté, dupliquer la ligne du milieu
deux ou trois fois pour de meilleurs résultats.

\section{Bring It On}

\subsection{Héritage}

Avoir des pattes, qui plus est en avoir quatre, ce n'est pas le propre des
chats: c'est également le cas des chiens, et en fait de tous les quadrupèdes.

Retirez les pattes de votre classe \verb|cat|; vous allez placer cet attribut
dans une classe mère \verb|quadrupede| dont vous allez hériter. Son interface:

\begin{verbatim}
class quadrupede : object
  val mutable legs : int
end
\end{verbatim}

N'oubliez pas de changer votre classe \verb|cat| pour mettre en place la
relation d'héritage. (Note: ceci ne change pas son interface) Si vous n'avez
pas pensé à retirer l'attribut \verb|legs| de votre chat, vous aurez cet
avertissement:

\begin{verbatim}
Warning 13: the instance variable legs is overridden.
\end{verbatim}

\subsection{Méthodes virtuelles}

En programmation orientée objet, une fonction virtuelle est une fonction
définie dans une classe, et qui est destinée à être redéfinie dans les
classes qui en hérite.

Modifiez votre classe \verb|quadrupede| pour lui ajouter une méthode virtuelle
\verb|species|, qui vous implémenterez ensuite dans votre classe dérivée
\verb|cat|, et qui retourne une chaîne de cractères qui décrit l'espèce de
votre bestiole. La nouvelle interface de \verb|quadrupede| est:

\begin{verbatim}
class virtual quadrupede : object
  val mutable legs : int
  method virtual species : unit -> string
end
\end{verbatim}

Faites également une classe \verb|dog| qui hérite de quadrupède, et implémente
bien entendu cette méthode \verb|species|.

\subsection{Sous-typage\protect\footnote{There is no actual subtyping in
\textsc{OCaml}
(\protect\url{http://brianmckenna.org/blog/row\_polymorphism\_isnt\_subtyping}), but no one
seems to care.}}

In programming language theory, \textbf{subtyping} (also \textbf{subtype
polymorphism} or
\textbf{inclusion polymorphism}) is a form of type polymorphism in which a
\textbf{subtype} is a
datatype that is related to another datatype (the \textbf{supertype}) by some
notion of \textit{substitutability}, meaning that program elements, typically
subroutines or functions, written to operate on elements of the supertype can
also operate on elements of the subtype.

In object-oriented programming the term `polymorphism' is commonly used to
refer solely to this subtype polymorphism, while the techniques of parametric
polymorphism would be considered \textit{generic programming}.

In the branch of mathematical logic known as type theory, \textbf{System
F\textsubscript{<:}},
pronounced "F-sub", is an extension of system F with subtyping. System
F\textsubscript{<:} has
been of central importance to programming language theory since the 1980s
because the core of functional programming languages, like those in the ML
family, support both parametric polymorphism and record subtyping, which can be
expressed in System F\textsubscript{<:}.\footnote{Computer \textbf{Science} is
about more than making shit compile}

\newpage

Tentons de faire une liste d'animaux:

\begin{verbatim}
type farm = quadrupede list
\end{verbatim}

Si on tente d'y aller naïvement, on échoue:

\begin{verbatim}
let li : farm = new cat :: new dog :: []
(* Error: This expression has type cat but an expression was expected of type
            quadrupede *)
\end{verbatim}

Il faut faire une \textit{coercion} explicite pour effectuer du sous-typage:

\begin{verbatim}
let li : farm = (new cat :> quadrupede) :: (new dog :> quadrupede) :: [] in
List.iter (fun c -> print_endline (c#species())) li
\end{verbatim}

Plus simplement:

\begin{verbatim}
# let x : quadrupede = new cat;;
(* Error *)

# let x :> quadrupede = new cat;;
val x : quadrupede = <obj>
\end{verbatim}

\section{Hurt Me Plenty}

Ajoutez une méthode \verb|fallout| qui incrémente le nombre de pattes de vos
chiens.

Modifiez les méthodes \verb|species| pour qu'elles donnent également le
nombre de pattes de votre animal:

\begin{verbatim}
method species () = string_of_int legs ^"-legged dog"
\end{verbatim}


\subsection{Design pattern: Visitor}

Si déporter des opérations contenues dans une classe vers une autre peut
sembler mauvais au sens POO, il y a de bonnes raisons pour le faire. En effet,
si ces opérations sont identiques pour chaque classe au lieu de dupliquer cette
méthode il est préférable de mettre ces opérations dans un visiteur
(centralisation de l'opération). Le visiteur utilisera ensuite les données
internes de chaque objet pour effectuer l'opération demandée.

En pratique, le modèle de conception visiteur est réalisé de la façon suivante
: chaque classe pouvant être « visitée » doit mettre à disposition une méthode
publique « accepter » prenant comme argument un objet du type « visiteur ». La
méthode « accepter » appellera la méthode « visite » de l'objet du type «
visiteur » avec pour argument l'objet visité. De cette manière, un objet
visiteur pourra connaître la référence de l'objet visité et appeler ses
méthodes publiques pour obtenir les données nécessaires au traitement à
effectuer (calcul, génération de rapport, etc.).

En programmation orientée objet et en génie logiciel, le patron de conception
comportemental visiteur est \textbf{une manière de séparer un algorithme d'une
structure de données}.

\subsection{Implémentation}

Nous shouhaitons ajouter une méthode \verb|accept| qui prend en argument un
visiteur à nos animaux. Or, le type \verb|visitor| n'est pas encore défini et,
contrairement à des langages tel le C, on ne peut pas faire de \textit{forward
declaration}. Il est donc nécessaire de paramétrer nos classes animales par ce
visiteur.

Le visiteur implémentera les méthodes \verb|visitCat| et \verb|visitDog|, qui
prennent respectivement en argument un chat et un chien. La méthode
\verb|accept| doit appeler ces fonctions sur les objets eux-mêmes.

\vspace{3mm}

Voici comment procéder:

\begin{verbatim}
class virtual ['visitor] quadrupede =
object
  ...
  method virtual accept : 'visitor -> unit
end

class ['visitor] cat =
object(self)
  inherit ['visitor] quadrupede
  ...
  method accept v = v#visitCat self
end

class ['visitor] dog =
object(self)
  inherit ['visitor] quadrupede
  ..
  method accept v = v#visitDog self
end
\end{verbatim}

Implémentez maintenant les visiteurs, qui dérivent de la classe \verb|visitor|
(qui définie les méthodes \verb|visitCat| et \verb|visitDog|) et qui prennent
en argument un objet de la classe qui va bien.

\begin{verbatim}
class virtual visitor =
object
  method virtual visitCat : visitor cat -> unit
  method virtual visitDog : visitor dog -> unit
end
\end{verbatim}

Le visiteur \verb|legsVisitor| (qui, comme tout visiteur, implémente les deux
méthodes \verb|visitCat| et \verb|visitDog|) va appeler la méthode qui modifie
le nombre de pattes de son argument.

Le visiteur \verb|speciesVisitor| va appeler \verb|species| de son argument.


\begin{verbatim}
class legsVisitor =
object
  inherit visitor
  method visitCat c = c#break_a_leg()
  method visitDog d = d#fallout()
end

class speciesVisitor =
object
  inherit visitor
  method visitCat c = print_endline (c#species())
  method visitDog d = print_endline (d#species())
end
\end{verbatim}

On peut maintenant tester:

\begin{verbatim}
let li : farm = (new cat :> quadrupedev) :: (new dog :> quadrupedev) :: (new
cat :> quadrupedev) :: [] in
List.iter (fun c -> c#accept (new legsVisitor)) li;
List.iter (fun c -> c#accept (new speciesVisitor)) li
\end{verbatim}

Affiche:
\begin{verbatim}
3-legged cat
5-legged dog
3-legged cat
\end{verbatim}

\end{document}
