
\documentclass[a4paper]{article}
\usepackage[utf8]{inputenc}
\usepackage[T1]{fontenc}
\usepackage[french]{babel}
\usepackage{fixltx2e}
\usepackage{nth}
\usepackage{listings}
\usepackage{tikz}
\usepackage{fancyhdr}
\usepackage[margin=1.0in]{geometry}
\usepackage{hyperref}

\pagestyle{fancy}
\lhead{{\sc{OCaml}}\\ {\sc TP-Spé} Traitement d'image - 11 oct. 2013}
\rhead{{\small \sc{GConfs}}\\ {\sc Epita}}
\rfoot{\includegraphics[width=0.15\linewidth]{gconfs.png}}

 \begin{document}
\begin{center}
  {\Large {\bf Le traitement d'image en
  \textsc{OCaml}}}
\end{center}

\bigskip

\section*{Introduction}

Ce soir (voir nuit pour les plus téméraires) nous allons nous intéresser au traitement d'images en \textsc{OCaml}, et c'est pour cela que 
vous êtes restés me direz-vous !\\
Tout d'abord laissez moi vous dire que durant ce TP nous allons principalement vous introduire les
 bases du traitement d'images afin qu'au final vous soyez capable de réaliser tout le nécéssaire
 pour le pré-traitement d'une image dans un OCR (en gros tout ce dont vous avez besoin pour votre première soutenance).\\
Le TP sera découpé en plusieurs petits exercices mettant en jeu des traitements d'images divers et variés.\\\\
Durant ce tp 2 armes seulement vous seront nécéssaires:
\begin{itemize}
\item Les bibliothèques SDL (normalement déjà installés sur vos racks)
\item La documentation de SDL (\url{ocamlsdl.sourceforce.net} partie Ressouces, puis Ocamldoc)
\end{itemize}

\section{Premiers pas avec SDL}
\subsection{Le chargement des images}
\subsection{Jouons avec les pixels}
\section{Les filtres basiques}
\subsection{Force rouge, force bleue ...}
\subsection{Nuance de gris}
\subsection{Inversion}
\section{Les filtres avancés}


\end{document}
