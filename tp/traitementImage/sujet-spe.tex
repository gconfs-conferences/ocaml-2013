\documentclass[a4paper]{article}
\usepackage[utf8]{inputenc}
\usepackage[T1]{fontenc}
\usepackage[french]{babel}
\usepackage{fixltx2e}
\usepackage{nth}
\usepackage{listings}
\usepackage{tikz}
\usepackage{fancyhdr}
\usepackage[margin=1.0in]{geometry}
\usepackage{hyperref}

\pagestyle{fancy}
\lhead{{\sc{OCaml}}\\ {\sc TP-Spé} Traitement d'image - 11 oct. 2013}
\rhead{{\small \sc{GConfs}}\\ {\sc Epita}}
\rfoot{\includegraphics[width=0.15\linewidth]{gconfs.png}}

 \begin{document}
\begin{center}
  {\Large {\bf Le traitement d'image en
  \textsc{OCaml}}}
\end{center}

\bigskip

\section*{Introduction}

Ce soir (voir nuit pour les plus téméraires) nous allons nous intéresser au traitement d'image en \textsc{OCaml}, et c'est pour cela que 
vous êtes restés me direz-vous !\\
Durant ce TP nous allons principalement introduire les
 bases du traitement d'images afin qu'au final vous soyez capable de réaliser quasiment tout le nécéssaire
 pour le pré-traitement d'une image dans un OCR.\\\smallskip
Le TP sera découpé en plusieurs petits exercices mettant en jeu des traitements d'images divers et variés.\\\\
Durant ce TP 2 armes seulement vous seront nécéssaires:
\begin{itemize}
\item Les bibliothèques SDL (normalement déjà installés sur vos racks)
\item La documentation de SDL (\url{http://ocamlsdl.sourceforge.net/doc/html})
\end{itemize}


\section{Premiers pas avec SDL}
\subsection{L'initialisation de SDL}
Avant de pouvoir commencer ce TP, la moindre des choses serait d'initialiser SDL :

\begin{verbatim}
  Sdl.init ['VIDEO]
\end{verbatim}

Notez que pour ce TP nous initialiserons SDL avec 'VIDEO et non 'EVERYTHING : nous n'avons besoin que de modifier des images donc seulement de manipuler des surfaces.


\subsection{Le chargement et la sauvegarde des images}
Avant toute chose, pouvoir ouvrir sur une image serait une bonne chose, mais comment charger une image pour l'utiliser dans mon code me direz-vous ?
Et bien c'est très simple, SDL intègre une fonction natif pour charger des images au format bitmap. Attendez ne partez pas ! Avec SDL il est également posible de charger du jpeg, du png voir même du tiff, cela grâce à un sous-module de SDL : Sdlloader.\\\smallskip
Ainsi les fonctions qui vous aiderons pour charger une image sont :

\begin{verbatim}
  val Sdlvideo.load_BMP : string -> surface
  val Sdlloader.load_image : string -> surface
\end{verbatim}

\noindent
Par contre pour la sauvegarde d'une image Sdl ne pourra gérer le format bitmap :

\begin{verbatim}
  val Sdlvideo.save_BMP : surface -> string -> unit
\end{verbatim}


\subsection{Jouons avec les pixels}

Maintenant que nous avons notre image essayons de la modifier. Abordons les fonctions qui traitent de la chose.\par\bigskip

La première chose à faire avant de manipuler une image comme on s'apprête à le faire, c'est de verrouiller la surface en memoire. Si cette fonction n'est pas utilisée SDL rencontrera de sérieux problèmes optimisation et prendra un certain temps à faire les modifications.

\begin{verbatim}
  val lock : surface -> unit
  val unlock : surface -> unit
\end{verbatim}

\noindent
Cela fait voici les fonctions permettantes de modifier les pixels (enfin...).

\begin{verbatim}
  val get_pixel_color : surface -> x:int -> y:int -> color
  val put_pixel_color : surface -> x:int -> y:int -> unit
\end{verbatim}

\section{Les filtres basiques}

Nous allons enfin pouvoir commencer la partie intéréssante !\\
Les fonctions de cette partie seront faites, sauf contre indication, selon le prototype suivant :

\begin{verbatim}
  val filtre : surface -> unit
\end{verbatim}


\subsection{Force rouge, force bleue ...}
Les premiers filtres vont consister en des filtres de couleurs :\\
\begin{itemize}
\item Écrivez un filtre qui ne garde que les pixels rouges.
\item Puis un filtre qui met une des composantes RGB (Red Blue Green) à zéro selon la string 
passée en paramètre ("blue" pour enlever toute trace de bleu)

\begin{verbatim}
val discard : surface -> string -> unit
\end{verbatim}

\item Ecrivez une fonction qui inverse toutes les couleurs de votre image (par exemple si la composante bleue est à 242 elle passe à 13)

\item Enfin faîtes un filtre noir ou blanc. Les pixels n'ont pas le droit d'être d'une autre couleur, à vous de trouver le juste milieu.

\end{itemize}
\subsection{Nuance de gris}

En premier lieu ecrivez une fonction qui fait la moyenne de chacune de vos composantes et qui remplace cette moyenne par les valeurs des comosantes.

Ensuite trouvez le moyen de faire un filtre de niveau de gris qui rende compte de la manaière dont l'oeil humain perçoit les composantes RGB (Cf: Wikipedia, Niveau de gris)

\section{Les filtres avancés}
\subsection{Les matrices de convolution}
Malgré le nom barbare que portent ces pauvres matrices, celles-ci nous sont bien utiles pour le traitement d'image.\\
Mais commençons par le commencement. Qu'est-ce qu'une matrice de convolution ?\\
En traitement d'image il existe une technique assez répendue qui consiste à créer une matrice pour chaque pixel plus les 8 pixels alentours.
C'est-à-dire que lorsque l'on traite un pixel, on utilise une matrice (ici 3x3) dont il est le centre.\\
Ensuite on effectue une opération avec une matrice de convolution (appelé aussi noyau). Cette opération consiste généralement 
à faire la multiplication de chacun des termes des 2 matrices respectives puis de sommer le tout pour attribuer cette valeure au pixel étudié.\par\smallskip
Grossièrement: 	\begin{math}
		P = \Sigma ( M(i,j) \times  N(i,j) )
		\end{math} \par\smallskip
Avec P le pixel, M la matrice initiale et N la matrice de convolution (noyau).\par\bigskip

Tout d'abord écrivez la fonction permettant de récuperer une matrice à partir d'un pixel.
(attention: Certains pixels n'ont pas forcément 8 voisins.)

\subsubsection{Filtre d'approximation}
A partir ce cet algorithme, écrivez une fonction qui pour chaque pixel calcule la moyenne de chaque composante des 9 pixels et donne ce code RGB au pixel étudié.
\subsubsection{Filtre de Tuckey}
Maintenant il s'agit de récuperer chacune des composantes RGB, de les mettre séparément dans des listes, 
vecteurs ou ce que vous souhaitez, puis puis donnez respectivement comme compossantes au pixel courant la médiane de chacune des couleurs.
\subsubsection{Convolution à volonté}
L'idée est de créer une fonction qui applique une matrice de convolution passée en paramètre comme filtre de l'image.

\section{Vous êtes sur la bonne voie !}
Vous êtes maintenant capable d'implémenter des filtres utiles (que l'on ne me dise pas que ce qui était avant n'était pas utile) et intéréssants.
Parmis eux nous vous conseillons de chercher:
\begin{itemize}
\item Les filtres de filtrages du bruit (Ex: Canny)
\item Les filtres de detection de contour (Ex: Sobel, Canny)
\item Opérateurs morphologiques (Cf: Wikipedia, Morphologie mathématique)
\item Le tramage (Ex: Bayer, Halftone, Floyd–Steinberg)
\end{itemize}

\end{document}
